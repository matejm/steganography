AES (Advanced Encryption Standard) je eden najbolj uporabljenih standardov za simetrično enkripcijo. Simetrična enkripcija pomeni, da imata pošiljatelj in prejemnik isti ključ, s katerim kriptirata oziroma dekriptirata sporočilo.

Algoritem sta razvila belgijska kriptografa Joan Daemen in Vincent Rijmen, ter ga poimenovala Rijndael. Pozneje je standard z manjšimi spremembami prevzela ameriška vlada kot naslednji standard po uporabi DES (Data Encryption Standard), saj so ključi DES postali prekratki in jih je bilo mogoče razbiti z močnejšimi računalniki.

AES je ``substitution-permutation network'', kar pomeni, da glavnino operacij, ki jih opravlja, predstavljajo različne zamenjave in permutacije bitov.

Poleg višje varnosti je glavna prednost standarda AES hitrost. Operacije, ki jih opravlja so nezahtevne, v nasprotju z asimetričnimi kriptiranji. AES je možno celo zapisati v vezje v procesorju, tako da lahko današnji običajni namizni računalniki kriptirajo AES tudi z več TB/s.

\subsection{Algoritem}
    Algoritem poenostavljeno sestoji iz štirih korakov, ki se naprej delijo na manjše korake.
    \begin{enumerate}
        \item \texttt{KeyExpansions} - razširitve ključa. Ključ, ki je vnaprej določene dolžine se razširi na več ključev, saj AES za vsak krog zahteva svoj ključ.
        \item \texttt{InitialRound} - dodajanje ključa. Nastavjo se začetna stanja s pomočjo posameznih delov ključa.
        \item \texttt{Rounds} - del, ki se večkrat ponovi, vsakokrat z novim ključem, ki smo ga razširili iz originalnega ključa. Če je ključ 128-biten se ponovi 10-krat, 12-krat pri 192-bitnih ključih in 14-krat pri 256-bitnih ključih.
            \begin{enumerate}
                \item \texttt{SubBytes} - preprosta substitucija znakov z uporabo tabele. Pri tem je pomembno, da ime tabela določene lastnosti, ki naredijo to preslikavo nelinearno, kar zelo oteži razbijanje šifre.
                \item \texttt{ShiftRows} - operacija na vrsticah trenutnega stanja. Vsak bit v neki vrstici se ciklično zamakne za neko število. Biti iz konca se premaknejo na začetek.
                \item \texttt{MixColumns} - korak, kjer se stolpci zamenjajo z drugimi stolpci. Vsa stanja v novem stolpcu so neposredno odvisna od vsakega posameznega stanja v prvotnem stolpcu. Če spremenimo en znak, se popolnoma spremeni celoten novi stolpec.
                \item \texttt{AddRoundKey} - korak skoraj enak postopku v InitialRound. Stanju se doda nov ključ, ki ustreza trenutnemu krogu, z operacijo \texttt{XOR} (ekskluzivni ali).
            \end{enumerate}
        \item \texttt{Final Round} - še zadnja ponovitev, ki je skoraj enaka vsem ostalim krogom, edina razlika je, da ne vsebuje koraka \texttt{MixColums}.\cite{wikipedia-aes}
    \end{enumerate}
