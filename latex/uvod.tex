Kmalu potem, ko sem se naučil programirati, me je začelo zanimati kriptiranje, varovanje podatkov in podobno. Pri kriptiranju pa je eden od ključnih problemov to, da ko želimo nekomu nekaj sporočiti, to vsi drugi opazijo. Kljub temu, da ne vedo o čem poteka pogovor, še vedno vedo kdo komunicira s kom. Steganografija pa odpravi točno ta problem. Sporočilo, ki ga pošljemo je zakrinkano, tako, da sploh ni opazno da želimo komu kaj sporočiti. \cite{wikipedia}

Odločil sem se, da bom kljub temu, da že obstaja več programov za steganografijo slik, naredil svoj program. Izbral sem si programski jezik Python, saj imam v njem daleč največ izkušenj v primerjavi z ostalimi programskimi jeziki.
