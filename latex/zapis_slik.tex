\label{zapisslik}
Slika je v računalniku zapisana tako, da je razdeljena na drobne kvadratke, imenovane piksli. Vsak piksel ima podatke kako močno in v katerih barvah žari. Za zapis se uporablja več različnih formatov, med katerimi so najbolj znani JPEG, PNG, BMP, GIF\ldots Med seboj se razlikujejo v različnih lastnostih, kot so način zgoščevanja, število možnih barv posameznega piksla\ldots

Osnovna zgradba vsakega formata sestoji iz:
\begin{description}
    \item [glave,] ki vsebuje glavne podatke, kot so velikost, barvna globina in kompresijska tehnika. 
    \item [slikovnih podatkov] oziroma niza pikslov. Podatki so lahko kompresirani ali nekompresirani.
    \item [polj za metapodatke] (metadata), kot so datum posnetka, avtor slike\ldots
\end{description} 

\subsection{Format PNG}
    PNG (Portable Network Graphics) je razmeroma nov slikovni format, ki je popularen predvsem na spletu.

    Format PNG uporablja brezizgubno kompresijo, kar nujno potrebujemo pri steganografiji, saj se zanašamo na to da bomo podatke lahko skrili tako, da jih človeško oko ne bo zaznalo. Izgubna kompresija pa izpušča podatke, ki jih človeško oko ne more zaznati, ter tako onemogoča pridobivanje skritega sporočila nazaj iz slike.

    Ena izmed prednosti formata PNG pred drugimi slikovnimi formati je podpora več različnih barvnih tabel. Poleg standardnega RGB (Red Blue Green) zapisa podpira tudi RGBA (Red Green Blue Alpha) zapis, katerega bistvena razlika je dodaten alpha kanal. Le-ta se navadno uporablja za prosojnost slik. Če je vrednost kanala 0\% je piksel popolnoma prosojen, če pa je vrednost 100\% pa je piksel enak običajnim pikslom.

    V primeru steganografije dodaten kanal veliko pripomore, saj lahko v sliko skrijemo kar $\frac{1}{4}$ več informacij, kot v zapisu RGB (Glej poglavje: \nameref{steganografijaslik}).